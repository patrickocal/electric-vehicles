\documentclass[12pt, draft]{article} \usepackage[shortlabels]{enumitem}
\usepackage{url}
\usepackage{amsthm}
\newtheorem*{example*}{Example}
\begin{document}
\begin{centering}
  \textsc{Submission to the Australian Government Department of Climate Change,
  Energy, the Environment and Water\\ in response to the:\\
  National Electric Vehicle Strategy consultation paper}\\
\vskip10pt
  \emph{Patrick O'Callaghan\\
  The Australian Institute for Business and Economics\\
  at The University of Queensland}\\
\end{centering}
\vskip20pt

We structure the present submission by responding directly to the
questions posed in the (consultation) paper. The first three questions are: 

\begin{enumerate}
 
  \item \emph{Do you agree with the objectives and do you think
      they will achieve our proposed goals? Are there other objectives we
      should consider?}

  \item \emph{What are the implications if other countries accelerate EV
    uptake faster than Australia?}

  \item \emph{What are suitable indicators to measure if we are on track to
    achieve our goals and objectives?}

\end{enumerate}

Basic economic principles tell us that the host of policies that we choose to
implement in Australia should result in market participants internalising the
social cost of their actions in relation to climate change. In this setting,
one difficulty that arises is that, individually, our actions are
insignificant, even though collectively we change the climate.  This difficulty
is nowhere more evident than in the case of global vehicle emissions.  Even at
the Australian level, since 1.1\% (2019 data) of global vehicle sales are in
Australia, our actions will not have a major impact on climate change. Our
policies are instead guided by the principle that we wish to be active
participants in the global transition to net zero, not only because it is the
right thing to do, but also because our economy may stand to benefit in a
variety of ways many of which are listed in section 1.2 of the paper.

The potential benefits of the transition that are most closely related to the
EVs are the following: health benefits of lower emissions; increased supply of
suitable minerals for batteries; a new manufacturing base for batteries and
other vehicle components that vertically integrates the EV supply chain.
\begin{example*}
 
  An example of successful vertical integration and our capacity to supply, not
  only raw but, \emph{refined materials} for batteries can be found in
  Gladstone, Queensland. In 2020, Alpha HPA commenced a project to produce high
  purity alumina (aluminium oxide) for batteries. The new process uses standard
  alumina that has been refined in Gladstone, Central Queensland since the
  1960s when deposits of bauxite (the raw material for alumina) were discovered
  in Weipa, Far North Queensland.  Moreover, this alumina uses Bauxite that is
  mined in Weipa, North Queensland.

\end{example*} 

A second, no less important, basic economic principle is that, in transforming
our economy, we should strive to do so in the most economically efficient way.
The economically efficient way to achieve global net-zero emissions by 2050 is
via a uniform carbon price
(UCP).\footnote{\url{https://www.nature.com/articles/526315a}\\
\url{https://www.frbsf.org/economic-research/wp-content/uploads/sites/4/gollier_reguant_climate_chapter.pdf}}
A UCP would provide an essential signal to guide economic agents to act in the
collective interest. It would also reduce the need for sector and
region-specific subsidies, taxes and regulations that distort agent behaviour
and raise the cost of transition. In the absence of deeper international
coordination, a global UCP is unattainable in the near future. Nonetheless, the
over-arching goal of Australian government policy should still be to coordinate
across divisions, be they departmental, sectoral or regional, and strive to
implement policies that are congruent with a \emph{national UCP}.

\paragraph{In relation to Q1:} In a nutshell: the main goal/objective should be
to ensure EV-related policies put us on a path towards a national UCP.

\paragraph{In relation to Q2:} Battery technology is unlike other green
technologies (such as solar panels and wind turbines) in that

\begin{itemize}
 
  \item will evolve substantially over the coming decades if it is to support
    not only EVs, but also houses, industry and the grid itself. 
 
  \item Australia has an existing comparative advantage in both the raw (and
    the refined) materials for batteries.

\end{itemize} 
   
This presents a significant opportunity, and although Australia would not
be a first-mover, innovation that leads to vertical integration in the
future has the potential to, not only generate new jobs, but also secure
existing jobs in the same supply chain (as highlighted in the alumina
example above.

\paragraph{In relation to Q3:} a national UCP would provide the most
appropriate normative benchmark relative to which we should seek to measure and
judge our policies. Identifying suitable indicators to benchmark current and
future policies to a national UCP would be easier once an estimate of the
optimal national UCP is established. Agreement on an optimal national UCP based
on research is a necessary first step.

\vskip20pt

Against this backdrop, I now briefly address the three proposed objectives on
page 6 (and sections $3.1, 3.2$ and $3.3$)
of the consultation paper.

\paragraph{Demand objective: encourage rapid increase in demand for EVs.} If
policies are to encourage a rapid increase demand, then they also need to
encourage an equally rapid increase in supply.  Otherwise, this objective
carries the risk of increasing prices of EVs relative to ICEs: precisely the
price signal we wish to avoid sending to future buyers.  Since only a little
over one percent of vehicles produced globally are sold to Australia, a rapid
increase in Australian demand \emph{alone} is unlikely to have a significant
and lasting impact on manufacturers' prices of EVs.\footnote{In contrast, if
Australian dealers are free to set local prices, then local prices are likely
to be sensitive to any rapid increases in demand.} Of course, we are not alone
in seeking rapid increases in demand, so, in the absence of international
coordination, such policies will collectively lead to large price increases and
outsized profits for manufacturers. Over time at the global level, this will
encourage new and existing manufacturers of EVs to expand production. The
increase in supply should finally lead to a fall in prices and allow us to
achieve the goal of increasing demand. What is not clear is whether this global
process would be rapid.

In light of this, a more direct approach would be to encourage new and existing
manufacturers to increase their rate of production and innovation. Since
Australia has the potential to play a major role in the supply chain for
batteries, policy should focus attention here.

\paragraph{Supply objective: increase supply of affordable and accessible EVs
to meet demand across all segments.} I broadly agree with this objective. I
recommend that this is achieved by

\begin{enumerate}[(a)]

\item
opening up the market to independent importers of new and used cars: this would
encourage new market entrants and competition to benefit of lower EV prices.

\item
bringing Australian fuel efficiency standards into line with other major
markets so as to avoid distortions.

\end{enumerate}

The main potential adverse effect of such a policy is the following. If car
prices (ICE and EV) fall relative to energy-efficient modes of collective
transport, then this might lead to an increase in car usage and an inefficient
increase in energy. The resulting upward pressure on electricity prices might
then slow the process of electrification more broadly.

\paragraph{Infrastructure objective: Establish the systems and infrastructure
to enable the rapid uptake of EVs.}

This is the policy with the clearest benefits. It will reduce the cost and
increase the benefits of EV ownership.

\end{document}
