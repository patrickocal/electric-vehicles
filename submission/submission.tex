\documentclass[12pt, draft]{article} \usepackage[shortlabels]{enumitem}
\usepackage{url}
\usepackage{amsthm}
\newtheorem*{example*}{Example}
\begin{document}
\begin{centering}
  \textsc{Submission to the Australian Government Department of Climate Change,
  Energy, the Environment and Water\\ in response to the:\\
  National Electric Vehicle Strategy consultation paper}\\
\vskip10pt
  \emph{Flavio Menezes and Patrick O'Callaghan\\
  The Australian Institute for Business and Economics}\\
\end{centering}
\vskip20pt

We structure the present submission by responding directly to the
questions posed in the (consultation) paper. The first three questions are: 

\begin{enumerate}
 
  \item \emph{Do you agree with the objectives and do you think
      they will achieve our proposed goals? Are there other objectives we
      should consider?}

  \item \emph{What are the implications if other countries accelerate EV
    uptake faster than Australia?}

  \item \emph{What are suitable indicators to measure if we are on track to
    achieve our goals and objectives?}

\end{enumerate}

Basic economic principles tell us that the host of policies that we choose to
implement in Australia should result in market participants internalising the
social cost of their actions in relation to climate change. In this setting,
one difficulty that arises is that, individually, our actions are
insignificant, even though collectively we change the climate.  This difficulty
is nowhere more evident than in the case of global vehicle emissions.  Even at
the Australian level, since 1.1\% (2019 data) of global vehicle sales are in
Australia, our actions will not have a major impact on climate change. Our
policies are instead guided by the principle that we wish to be active
participants in the global transition to net zero, not only because it is the
right thing to do, but also because our economy may stand to benefit in a
variety of ways many of which are listed in section 1.2 of the paper.

The potential benefits of the transition that are most closely related to the
EVs are the following: health benefits of lower emissions; increased supply of
suitable minerals for batteries; a new manufacturing base for batteries and
other vehicle components that vertically integrates the EV supply chain.
\begin{example*}
 
  An example of successful vertical integration and our capacity to supply, not
  only raw but, \emph{refined materials} for batteries can be found in
  Gladstone, Queensland. In 2020, Alpha HPA commenced a project to produce high
  purity alumina (aluminium oxide) for batteries. The new process uses standard
  alumina that has been refined in Gladstone, Central Queensland since the
  1960s when deposits of bauxite (the raw material for alumina) were discovered
  in Weipa, Far North Queensland.  Moreover, this alumina uses Bauxite that is
  mined in Weipa, North Queensland.

\end{example*} 

A second, no less important, basic economic principle is that, in transforming
our economy, we should strive to do so in the most economically efficient way.
The economically efficient way to achieve global net-zero emissions by 2050 is
via a uniform carbon price
(UCP).\footnote{\url{https://www.nature.com/articles/526315a}\\
\url{https://www.frbsf.org/economic-research/wp-content/uploads/sites/4/gollier_reguant_climate_chapter.pdf}}
A UCP would provide an essential signal to guide economic agents to act in the
collective interest. It would also reduce the need for sector and
region-specific subsidies, taxes and regulations that distort agent behaviour
and raise the cost of transition. In the absence of deeper international
coordination, a global UCP is unattainable in the near future. Nonetheless, the
over-arching goal of Australian government policy should still be to coordinate
across divisions, be they departmental, sectoral or regional, and strive to
implement policies that are congruent with a \emph{national UCP}.

\paragraph{In relation to Q1:} In a nutshell: the main goal/objective should be
to ensure EV-related policies put us on a path towards a national UCP.

\paragraph{In relation to Q2:} Battery technology is unlike other green
technologies (such as solar panels and wind turbines) in that

\begin{itemize}
 
  \item will evolve substantially over the coming decades if it is to support
    not only EVs, but also houses, industry and the grid itself. 
 
  \item Australia has an existing comparative advantage in both the raw (and
    the refined) materials for batteries.

\end{itemize} 
   
This presents a significant opportunity, and although Australia would not
be a first-mover, innovation that leads to vertical integration in the
future has the potential to, not only generate new jobs, but also secure
existing jobs in the same supply chain (as highlighted in the alumina
example above.

Below, we discuss further the pricing
difficulties that may arise in the absence of international coordination on
policies that rapidly increase demand for EVs.

\paragraph{In relation to Q3:} a national UCP would provide the most
appropriate normative benchmark relative to which we should seek to measure and
judge our policies. Identifying suitable indicators to benchmark current and
future policies to a national UCP would be easier once an estimate of the
optimal national UCP is established. Agreement on an optimal national UCP based
on research is a necessary first step.

\vskip20pt

Against this backdrop, we now address the three proposed objectives on page 6
(and associated questions $4$ to $20$ in sections $3.1, 3.2$ and $3.3$) of
the consultation paper.

\paragraph{Demand objective: encourage rapid increase in demand for EVs.}
This objective will face the obstacle of increasing EV prices in the short
term.

If policies are to encourage a rapid increase demand, then they also need to
encourage an equally rapid increase in supply.  Otherwise, this objective
carries the risk of increasing prices of EVs relative to ICEs: precisely the
price signal we wish to avoid sending to future buyers.  Since only a little
over one percent of vehicles produced globally are sold to Australia, a rapid
increase in Australian demand alone is unlikely to have a significant impact on
manufacturers' prices of EVs.\footnote{In contrast, if Australian dealers are
free to set local prices, then local prices are likely to be sensitive to any
rapid increases in demand.} Of course, we are not alone in seeking rapid
increases in demand, so, in the absence of international coordination, such
policies will collectively lead to large price increases and outsized profits
for manufacturers. Over time at the global level, this will encourage new and
existing manufacturers of EVs to expand production. The increase in supply
should finally lead to a fall in prices and allow us to achieve the goal of
increasing demand. What is not clear is whether this global process would be
rapid.

In light of this, a more direct approach would be to encourage new and existing
manufacturers to increase their rate of production and innovation. Since
Australia is unlikely to play a major role in the manufacturing and 

\paragraph{Supply objective: increase supply of affordable and accessible EVs
to meet demand across all segments.} We broadly agree with this objective. We
recommend that this is achieved by

\begin{enumerate}[(a)]

\item
opening up the market to independent importers of new and used cars: this would
encourage new market entrants and competition to benefit of lower EV prices.

\item
bringing Australian fuel efficiency standards into line with other major
markets so as to avoid distortions.

\end{enumerate}

The main potential adverse effect of such a policy is the following. If car
prices (ICE and EV) fall relative to energy-efficient modes of collective
transport, then this might lead to an increase in car usage and an inefficient
increase in energy. The resulting upward pressure on electricity prices might
then slow the process of electrification more broadly.

\paragraph{Infrastructure objective: Establish the systems and infrastructure
to enable the rapid uptake of EVs.}


In the absence of a coordinated international approach, the most effective way
to increase demand is to subsidise purchases of EVs.  Such subsidies carry
additional benefits in markets with domestic production (Zheng et al). This is
because the subsidy surplus flows, not only to the consumer, but also to the
manufacturer. In the case of EVs, much of the surplus will be passed on to
foreign manufacturers.


Furthermore, coordination is needed to ensure that there are no
major knock-on effects outside the sector. We will discuss further the
conditions under which it is possible to achieve an increase in Australian
demand without significant increases to EV prices below.

\paragraph{Objective:
    }




\emph{In achieving the objectives of the Strategy, we will address barriers to EV
uptake such as:
}

\paragraph{Barrier to uptake: Limited availability of affordable EV models
across all vehicle types}

\paragraph{Range anxiety due to gaps in EV charging networks and hydrogen
refuelling infrastructure }

Information for consumers.

\emph{4. Are there other measures by governments and industry that could increase
affordability and accessibility of EVs to help drive demand?}

\emph{5. Over what timeframe should we be incentivising low emission vehicles
as we transition to zero emission vehicles?}

\emph{6. What information could help increase demand and is Government or industry
best placed to inform Australians about EVs?}

We are seeking views on how vehicle fuel efficiency standards could be
implemented in Australia. If these standards are implemented, they will need to
be designed specifically for Australia. However, evidence also suggests that
standards that lack ambition will continue to leave Australia at the back of
the queue for cheaper, cleaner new vehicles. Feedback is sought on options for
a robust model. We will draw from the experience in other markets and consider
Australia’s unique circumstances.

Vehicle fuel efficiency standards need to be:

\paragraph{Effective in reducing transport emissions}

\paragraph{Equitable so all Australians can access the vehicles they need for work and
leisure}

\paragraph{Transparent and well explained to avoid unintended consequences}

\paragraph{Credible and robust by drawing on expert analysis and experience}

\paragraph{Enable vehicles with the best emissions and safety technology to be available
to Australians.} 

Initially, we are seeking views on:

\emph{7. Are vehicle fuel efficiency standards an effective mechanism to reduce
passenger and light commercial fleet emissions?}

\emph{8. Would vehicle fuel efficiency standards incentivise global manufacturers to
send EVs and lower emission vehicles to Australia?}

\emph{9. In addition to vehicle fuel efficiency standards for passenger and light
commercial vehicles, would vehicle fuel efficiency standards be an appropriate
mechanism to increase the supply of heavy vehicle classes to Australia?}

\emph{10. What design features should the Government consider in more detail for vehicle fuel efficiency standards, including level of ambition, who they should
apply to, commencement date, penalties and enforcement?}

\emph{11. What policies and/or industry actions could complement vehicle fuel
efficiency standards to help increase supply of EVs to Australia and electrify
the Australian fleet?}

\emph{12. Do we need different measures to ensure all segments of the road transport
sector are able to reduce emissions and, if so, what government and industry
measures might well support the uptake of electric bikes, micro-mobility and
motorbikes?}

\emph{13. How could we best increase the number of affordable second hand EVs?}

\emph{14. Should the Government consider ways to increase the supply of second hand EVs independently imported to the Australian market? Could the safety and
consumer risks of this approach be mitigated?}

\emph{15. What actions can governments and industry take to strengthen our
competitiveness and innovate across the full lifecycle of the EV value chain?}

\emph{16. How can we expand our existing domestic heavy vehicle manufacturing and
assembly capability?}

\emph{17. Is it viable to extend Australian domestic manufacturing and assembly
capability to other vehicle classes?}

\emph{18. Are there other proposals that could help drive demand for EVs and provide
a revenue source to help fund road infrastructure?}

\emph{19. What more needs to be done nationally to ensure we deliver a nationally
comprehensive framework for EVs?}

\emph{20. How can we best make sure all Australians get access to the opportunities
and benefits from the transition?}

\end{document}
