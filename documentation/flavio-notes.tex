Some very disorganised notes for our meeting today:

 

3.1 – encourage rapid increases of demand for EVs

 

    Make the general point that, given the supply shortages, initiatives to
    increase demand will push prices up in the short-run

        For example, the 75\% target will increase prices in the short run –
        with the benefit of putting downward pressure on prices of used EVs in
        the medium run – but there are other policies that will work on supply
        that will likely have a much higher benefit -cost ratio (see comments
        on 3.2). 
       
       Recent Research on the uptake of EVs in China (see my paper attached)
       shows:

        Price subsidies explain a large fraction of the significant increase in
        the private sales of EVs in China – however, as we don’t have local
        manufacturing of EVs, price subsidies make little sense in Australia –
        however, this strong supports the exemption from FBT and
        reduction/elimination of  import tariffs, stamp duty and registration
        charges.

        This research also shows that licence restrictions on ICEs have a small
        impact (e.g., limiting the number of car licences, or auctioning off
        the licences as in Singapore and in some Chinese cities) on private
        sales of EVs.

        Availability of charging stations also has a positive impact on sales –
        of the demand inducing measures, this is likely to have the largest
        benefit – cost ration – as it is impact is likely to be sustained over
        time.

 

        encourage increases in supply

    The rationale provided in the consultation paper for introducing fuel
    efficiency standards is uncontroversial. However, we may be placing too
    much faith in the ability of fuel efficiency standards to increase supply
    of EVs in Australia. We are a small buyer of cars (say 1 out of 65 million
    sold globally). Manufacturers would likely favour larger markets

    Instead, we should relax the regulation that limits the independent (from
    manufacturers) importation of new and used EVs. Currently, only those are
    limited to vehicles where the manufacturer is not selling that vehicle in
    Australia. The argument in the paper is that a reduction in regulation in
    this area can potentially compromise vehicle quality standards and road
    safety. Also, the paper expresses a concern that such vehicles are not
    covered by the same warranty and recall provisions provided by the original
    manufacturers.

        We have difficulties in understanding how a car that is considered safe
        to drive say in the UK, wouldn’t be safe to drive in Australia.

        . We should be more ambitious:

            Allowing independent importation would increase competition in the
            sale of new (and consequently used) cars overnight. Manufacturers
            would be competing against themselves and would have to reconsider
            how they price their supply and pricing strategies for the
            Australia market.  Concerns about warranties and services are
            misplaced in our view.  If this is a concern, the government can
            impose requirements on manufacturers to provide the same warrant
            regardless whether the sale is independent or through a dealer.
            Allowing independent importation would promote the development of a
            skilled workforce to service the cars and also potentially allow
            the development of local manufacturing of EV components. 

 

        establishing the systems and infrastructure

 

    Important to establish a nationally consistent approach – for example, for
    road charging. This means agreeing on principles. For example, at the
    moment, important to get policies for which extensive and intensive margins
    work in the right direction. Extensive margin is more important for uptake
    of EVs. For example, Vic policy of charging EV users per km is ok in the
    sense that it promotes cost recovery for the use of the road (intensive
    margin), but goes the other way on the extensive margin (does not encourage
    the sale of EVs vs ICEs. – so no effect on pollution).

    Congestion charging is the way to go – as while discourages both EV and
    ICEs during peak, but reduces pollution overall. 
